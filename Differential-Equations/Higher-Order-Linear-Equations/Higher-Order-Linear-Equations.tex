\documentclass{article}

\usepackage{amsmath}
\usepackage{amssymb}
\usepackage{amsthm}

\newtheorem{Thm}{Theorem}

\begin{document}
	
\author{Logan Grosz}
\title{Journal Template}
\date{\today}

\maketitle

\begin{abstract}
	Abstract goes here...
\end{abstract}

\section{Declarations}

$variable$; variable description; $variable\quad domain\quad and\quad range,\quad if\quad applicable$

\section{Rule}
\section{Pre-Derivation and Theorems}
	\begin{Thm}
		General n\textsuperscript{th} order differential equation:\\
		\begin{gather*}
			a_ny^{(n)}+a_{n-1}y^{(n-1)}+\dots+a_2y''+a_1y'+a_0y=g(x)\\
			(a_n,a_{n-1},\dots a_1,a_0)\in k
		\end{gather*}
	If \(g(x)=0\), differential equation is homogeneous.\\
	If \(g(x)\neq0\), differential equations is not homogeneous.
	\end{Thm}

	\begin{Thm}[]
		Let $y_1,y_2,\dots,y_k$ be solutions of the n\textsuperscript{th} order equation on I. Then, per linear combination, $y=c_1y_1+C_2y_2+\dots+c_ky_k$ is also a solution where $c_1,c_2,...c_k$ are arbitrary constants.\\
		
		\noindent Linear Dependence and Independence\\
		
		\noindent Set theory here\\
		
		\noindent Id Est:\\
		\noindent $f_1,f_2,\dots,f_n$ are said to be linearly dependent on I, if there exists a set of constants $k_1,k_2,\dots,k_n$, that aren't all zero, such that $k_1f_1+k_2f_2+\dots+k_nf_n=0$ for all x in I.\\
		
		\noindent$f_1,f_2,...f_n$ are linearly independent if they are not linearly dependent.
	\end{Thm}
\section{Derivation}

Derivation goes here

\section{Exempli Gratia}

Examples of important instances

\end{document}