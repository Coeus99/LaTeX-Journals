\documentclass{article}

\usepackage{amsmath}
\usepackage{amssymb}

\usepackage{tikz}
\usepackage{pgfplots}

\begin{document}
	
\author{Logan Grosz}
\title{Separable Equations}
\date{\today}

\maketitle

\begin{abstract}
	Abstract goes here...
\end{abstract}

\section{Declarations}

$\frac{dy}{dx}\triangleq y'$; Differential and prime notation equivalence;\\
$p(y)=\frac{1}{h(y)}$;P is the reciprocal of H, for neatness' sake;

\section{Rule}
$$\dfrac{dy}{dx}=g(x)\,h(y)$$\\
$$\implies\int p(y)\,dy=\int g(x)\,dx$$

\section{Pre-Derivation}
This rule only holds true if a differential equation is of first order and holds the following form, that being the product of two functions, one of x and one of y. See below...\\

$$\dfrac{dy}{dx}=g(x)\,h(y)$$

\section{Derivation}

\begin{align*}
	\dfrac{dy}{dx}&=g(x)\,h(y)\\
	\dfrac{1}{h(y)}\dfrac{dy}{dx}&=g(x)\\
	\text{Let }p(y)&=\dfrac{1}{h(y)},\,h(y)\neq 0\\
	\therefore\dfrac{dy}{dx}p(y)&=g(x)\\\\
	\text{Assume }y&=\phi(x),\text{ the solution.}\\
	\dfrac{dy}{dx}&=\phi '(x)\\\\
	p(\phi(x))\phi '(x)&=g(x)\\
	\int p(\phi(x))\phi '(x)dx&=\int g(x) dx\\\\
	\text{But since }y&=\phi(x)\\
	\implies dy&=\phi '(x)dx\\\\
	\therefore \int p(y)\,dy&=\int g(x)\,dx
\end{align*}

\section{Exempli Gratia}

\subsection{Solution branches with initial value}

\begin{align*}
	\dfrac{dy}{dx}&=\dfrac{2x}{y+x^2y},\,y(0)=-2\\
	y\,dy&=\dfrac{2x}{x^2}dx\\
	\int y\,dy&=\int\dfrac{2x}{1+x^2}dx\\
	\dfrac{y^2}{2}&=ln|1+x^2|+C\\
	\implies\dfrac{y^2}{2}&=ln(1+x^2)+C\\
	y&=\pm\sqrt{2\,ln(1+x^2)+C}\\\\
	y(0)&=2\implies x=0,\,y=-2\\
	\text{using }y^2&=2\,ln(1+x^2)+C\\
	4&=2\,ln(1+0)+C\\
	\implies C&=4\\
\end{align*}
\begin{center}
\begin{tikzpicture}
	\begin{axis}[
	xlabel=$x$,
	ylabel=$y$,
	axis y line=middle,
	axis x line=middle,
	]
	\addplot[mark=none]{(2*ln(1+x^2)+4)^(1/2)};
	\addplot[mark=none]{-(2*ln(1+x^2)+4)^(1/2)};
	\addplot[mark=*] coordinates{(0,-2)};
	\end{axis}
\end{tikzpicture}
\end{center}
The bottom branch is the solution to the given differential equation because it contains the point. No two discontinuous branches can be solutions at the same time.

\end{document}